\chapter{Sensor fusion}

A sensor fusion is a method of blending multiple sensors' reading to obtain a desired estimation. The methods involve merging information from a variety of sensors, cross-checking as well as tracking and removing floating error. They utilize partial and redundant information to estimate complete estimation.\\

There are many methods of sensor fusion in the application of determining position and orientation \cite{uav}. The classic orientation's estimation methods, that are based on gyroscope, accelerometer, are Complementary filter method \cite{complementary}, Direction Cosine Matrix method \cite{dcm} and Madgwick filter method \cite{madgwick}. \\

Complementary filter method involves estimating orientation two ways: incrementally and absolutely. An incremental estimation means integrating gyroscope readings to rotate from a previous moment. The disadvantage of incremental estimation is that bias is also integrated. Besides, accelerometer and magnetometer readings are used to estimate an absolute orientation, assuming that accelerometer reading vector points to down and gyroscope reading vector points to north. It is not perfect assumption as the accelerometer measures all other accelerations too. Also, the magnetometer readings require declination and inclination correction. In result, the absolute estimation is very noisy. The main idea of the method is to filter the incremental esimation with high pass filter that remove bias, and filter the absolute estimation with low pass filter that remove noise. The filter are selected to be complementary, so the sum of estimations covers the full bandwidth.\\

Direction Cosine Matrix (DCM) method is also based on incremental model with drift correction.
To correct drift DCM involves using PI controller. The method is strongly linked with rotation matrix's properties. Moreover, if velocity source is present, the method can explicitly compensate acceleration that is not gravitational. \\

Madgwick filter method is an algorithm developed by Sebastian Madgwick. The method uses a quaternion representation and structural definition, easy to implement in low-level languages. Unlike traditional methods like the Kalman filter, the Madgwick filter is highly efficient computationally, making it suitable for real-time applications even on resource-constrained platforms. However, it has closed structure, that makes it less readable. Quoting the description of the Python filter implementation \cite{madgwick2}: \textit{The filter employs a quaternion representation of orientation to describe the nature of orientations in three-dimensions and is not subject to the singularities associated with an Euler angle representation, allowing accelerometer and magnetometer data to be used in an analytically derived and optimised gradient-descent algorithm to compute the direction of the gyroscope measurement error as a quaternion derivative.}\\

All of presented method so far enable orientation estimation only and are hardly modifiable. This results in the need to find more sophisticated methods. For many years the most popular and universal method has been the use of the Kalman filter \cite{ekf_poor}.

\section{An Kalman Filter}

\todo[inline, color=green]{
	This section is directly taken and translated from my bachelor thesis. Should it be marked somehow?
}

The Kalman filter is a mathematical estimation method for efficiently tracking the dynamic state of a system and filtering out measurement data. It is based on modeling the system with state equations and measurement equations. The equations of state describe the evolution of the system state over time, while the measurement equations represent the relationship between the system state and the measurement data. Mathematically, this can be represented by equations (\ref{kf1} - \ref{kf2}).

\begin{equation}
	\bm{x}_k = A \cdot \bm{x}_{k-1} + B \cdot \bm{u}_k + \bm{w}_k
	\label{kf1}
\end{equation}

\begin{equation}
	\bm{z}_k = H \cdot \bm{x}_k + \bm{v}_k
	\label{kf2}
\end{equation}

where:
\begin{itemize}[noitemsep]
	\item $\bm{x}_k$ -- state vector at the time of k, 
	\item $A$ -- state matrix, 
	\item $B$  -- input matrix, 
	\item $\bm{u}_k$  -- control input vector, 
	\item $\bm{w}_k$ -- process noise vector, 
	\item $\bm{z}_k$  -- measurement vector at the time of k, 
	\item $H$ -- measurement matrix,
	\item $\bm{v}_k$ -- measurement noise vector.
\end{itemize}

The equations presented above are similar to the equation of state of a system with white noise added. The idea of the filter is based on finding the state, which from a statistical point of view is the most probable, for the recorded measurements. The disadvantage of the above notation is the requirement of linear state equation. Fortunately, the solution to this problem is provided by the use of Extended Kalman Filter (EKF), which state is presented in equations (\ref{ekf1} - \ref{ekf2}).

\begin{equation}
	\bm{x}_k =  f \left( \bm{x}_{k-1},  \bm{u}_k \right) + \bm{w}_k
	\label{ekf1}
\end{equation}

\begin{equation}
	\bm{z}_k = h \left(\bm{x}_k \right) + \bm{v}_k
	\label{ekf2}
\end{equation}


where:
\begin{itemize}
	\item $ f \left( \bm{x}_{k-1},  \bm{u}_k \right)$ -- state function, 
	\item $h \left(\bm{x}_k \right)$ -- measurement function.
\end{itemize}

The Kalman filter performs two main steps: prediction and correction (figure \ref{kf_diagram}). In the prediction step, the new state of the system and its covariance is predicted. In the correction step, based on the new measurements, an adjustment is made to the prediction, taking into account measurement errors and improving the estimation of the system state. The filter, especially in its extended version, is a major tool in the field of control and state's estimation, enabling systems to maintain the precision of measurements in dynamic conditions and improve the quality of system parameter estimation.\\


\begin{figure}[!h]
	\centering
	\begin{tikzpicture}[
		block/.style={rectangle, draw, text width=2cm, text centered, rounded corners, minimum height=1.5cm},
		arrow/.style={->, >=stealth, thick}
		]
		% Nodes
		\node (predict) [block] {Predict};
		\node (correct) [block, right=2cm of predict] {Correction};
		
		% Arrows
		\draw[arrow] (predict.east) -- ++(1.5cm,0) |- (correct.west) node[midway, above] {};
		\draw[arrow] (correct.west) -- ++(-1.5cm,0) |- (predict.east) node[midway, below] {};
	\end{tikzpicture}
	\caption{An Extended Kalman Filter}
	\label{kf_diagram}
\end{figure}

Prediction phase:
\[
\begin{aligned}
	\hat{x}_k^- & = f(\hat{x}_{k-1}, \bm{u}_k) \quad \text{(predicted state)}, \\
	P_k^- & = A_k P_{k-1} A_k^T + Q_k \quad \text{(predicted covariance)},
\end{aligned}
\]

where: $A_k = \frac{\partial f}{\partial x}\Bigr|_{\hat{x}_{k-1}, \bm{u}_k}$, and $Q_k$ is process noise's coveriance matrix.\\

Correction phase:
\[
\begin{aligned}
	&K_k = P_k^- H_k^T (H_k P_k^- H_k^T + R_k)^{-1} \quad \text{(Kalman's gain)}, \\
	&\hat{x}_k = \hat{x}_k^- + K_k(\bm{z}_k - h(\hat{x}_k^-)) \quad \text{(corrected state)}, \\
	&P_k = (I - K_k H_k) P_k^- \quad \text{(corrected covariance)},
\end{aligned}
\]

where: $H_k = \frac{\partial h}{\partial x}\Bigr|_{\hat{x}_k^-}$, and $R_k$ is measurement noise's covariance matrix.



\section{An Extended Kalman Filter as position and orientation estimator}

Application an Extended Kalman Filter to estimate position and orientation involves determination of state vector and state equation, specification measurement sources and their linkage with state. In prediction phase high sample rate sensors and sensor that measurements must be integrated should be used, like inertial sensor. The correction phase that can be processed rarely should include rest of sensors. In this section the filter's model will be presented.\\

In preparation, available sensors assigned into particular phase. The accelerometer and gyroscope's readings are used in prediction phase, as the reading are integrated. In correction phase external position source is used to correct position and accelerometer's readings are used to correct orientation.  Thus, the equation (\ref{u_vec}) presents the proposed control input vector and equation (\ref{h_vec}) presents measurement vector. Next state vector is set. The proposed state vector is presented in equation (\ref{state_vec}).

\begin{equation}
	\bm{u} = \begin{bmatrix}
		g_x & g_y & g_z & a_x & a_y & a_z
	\end{bmatrix}^T
	\label{u_vec}
\end{equation}

\begin{equation}
	\bm{z} = \begin{bmatrix}
		a_x & a_y & a_z & p_x & p_y & p_z
	\end{bmatrix}^T
	\label{h_vec}
\end{equation}

where:
\begin{itemize}
	\item $g_x$, $g_y$, $g_z$ -- 3 components of gyroscope's reading,
	\item $v_x$, $v_y$, $v_z$ -- 3 components of accelerometer's reading,
	\item $p_x$, $p_y$, $p_z$ -- 3 components of position from external provider.
\end{itemize}

\begin{equation}
	\bm{x} = \begin{bmatrix}
	x & y & z & v_x & v_y & v_z & q_0 & q_x & q_y & q_z & b_x & b_y & b_z & az & el
	\end{bmatrix}^T
	\label{state_vec}
\end{equation}

where:
\begin{itemize}
	\item $x$, $y$, $z$ -- 3 components of position,
	\item $v_x$, $v_y$, $v_z$ -- 3 components of velocity,
	\item $q_0$, $q_x$, $q_y$, $q_z$ -- quaternion of orientation,
	\item $b_x$, $b_y$, $b_z$ -- 3 components of gyroscope's bias,
	\item $az$ and $el$ -- azimuth and elevation of drawbar.
\end{itemize}

The bias and orientation of draw bar require an additional comment. The gyroscope's bias in state vector enable drift tracking and its compensation. Thanks to an Kalman Filter's properties the value of actual bias will be almost automatically calculated when the filter runs. The azimuth and elevation of drawbar enables utilizing external position sources even if they are not directly connected to system. In this implementation the position is given for point that is connected to system via link with fixed length. To simplify it is assumed that drawbar is connected by spherical joint to origin of moving coordinates system.\\

For the vectors so selected, the equation of state was determined. To increase readability, the state equation was split into equations (\ref{state_begin} - \ref{state_end}). The Jacobi matrix of the state function is calculated in equations (\ref{state_jacobi_begin} - \ref{state_jacobi_end}).




\begin{align}
	\bm{DCM} \left( \bm{q}_k \right) &=  \begin{bmatrix}
		q_0^2 + q_x^2 - q_y^2 - q_z^2 & 2(q_x q_y - q_0 q_z) & 2(q_x q_z + q_0 q_y) \\
		2(q_x q_y + q_0 q_z) & q_0^2 - q_x^2 + q_y^2 - q_z^2 & 2(q_y q_z - q_0 q_x) \\
		2(q_x q_z - q_0 q_y) & 2(q_y q_z + q_0 q_x) & q_0^2 - q_x^2 - q_y^2 + q_z^2 \\
	\end{bmatrix}
	\label{state_begin}
	\\
	\bm{S} \left( \bm{q}_k \right)  &= \begin{bmatrix}
		-q_x & -q_y & -q_z \\
		q_0 & -q_z & q_y \\
		q_z & q_0 & -q_x \\
		-q_y & q_x & q_0 \\
	\end{bmatrix}
	\\
	\bm{a}^W &= \bm{DCM} \left( \bm{q}_k \right) \begin{bmatrix}
		a_x \\ a_y \\ a_z 
	\end{bmatrix} - \begin{bmatrix}
		0 \\ 0 \\ g 
	\end{bmatrix}
\end{align}

\begin{align}
	\begin{bmatrix}
		x \\ y \\ z 
	\end{bmatrix}_{k+1}
	&= 
	\begin{bmatrix}
		x \\ y \\ z 
	\end{bmatrix}_{k}
	+ \Delta t
	\begin{bmatrix}
	v_x \\ v_y \\ v_z 
	\end{bmatrix}_{k}
	+ \frac{1}{2} (\Delta t)^2 \bm{a}^W
	\\
	\begin{bmatrix}
		v_x \\ v_y \\ v_z 
	\end{bmatrix}_{k+1}
	&= 
	\begin{bmatrix}
		v_x \\ v_y \\ v_z 
	\end{bmatrix}_{k}
	+ \Delta t \bm{a}^W
	\\
	\begin{bmatrix}
		q_0 \\ q_x \\ q_y \\ q_z 
	\end{bmatrix}_{k+1}
	&= 
	\begin{bmatrix}
		q_0 \\ q_x \\ q_y \\ q_z  
	\end{bmatrix}_{k}
	+ \frac{1}{2} \Delta t \bm{S} \left( \bm{q}_k \right) 
	\left(
	\begin{bmatrix}
		g_x \\ g_y \\ g_z
	\end{bmatrix}_{k}
	-
	\begin{bmatrix}
		b_x \\ b_y \\ b_z
	\end{bmatrix}_{k}
	\right)
	\\
	\begin{bmatrix}
		b_x \\ b_y \\ b_z
	\end{bmatrix}_{k+1}
	&= 
	\begin{bmatrix}
		b_x \\ b_y \\ b_z 
	\end{bmatrix}_{k}	
	\\
	\begin{bmatrix}
		az \\ el 
	\end{bmatrix}_{k+1}
	&= 
	\begin{bmatrix}
		az \\ el 
	\end{bmatrix}_{k}
	\label{state_end}
\end{align}


\begin{equation}
	\bm{D} \left( \bm{q}, \bm{a}\right)
	=
	2
	\begin{bmatrix}
		a_xq_0 + a_yq_z - a_zq_y & a_yq_0 - a_xq_z + a_zq_x & a_zq_0 + a_xq_y - a_yq_x \\
		a_xq_x + a_yq_y + a_zq_z & a_zq_0 + a_xq_y - a_yq_x & a_xq_z - a_xq_z - a_zq_x \\
		a_yq_x - a_xq_y - a_zq_0 & a_xq_x + a_yq_y + a_zq_z & a_xq_0 - a_yq_z + a_zq_y \\
		a_yq_0 - a_xq_z + a_zq_x & a_zq_y - a_yq_z - a_xq_0 & a_xq_x + a_yq_y + a_zq_z \\
	\end{bmatrix}^T
	\label{state_jacobi_begin}
\end{equation}

\begin{equation}
	\renewcommand{\arraystretch}{1.3}
	\bm{A} \left( \bm{x}, \bm{u} \right) = \frac{\partial \bm{f}}{\partial \bm{x}}
	= 
	\begin{bNiceArray}{c;c;c;c;c}
		\bm{I}_{3x3} & \Delta t \bm{I}_{3x3} & \frac{\Delta t ^2}{2}\ \bm{D} \left( \bm{q}, \bm{a}\right) & \bm{0}_{3x3} & \bm{0}_{3x2} \\
		\cdashedline
		\bm{0}_{3x3} & \bm{I}_{3x3} & \Delta t\ \bm{D} \left( \bm{q}, \bm{a}\right) & \bm{0}_{3x3} & \bm{0}_{3x2} \\
		\cdashedline
		\bm{0}_{4x3} & \bm{0}_{4x3} & \bm{I}_{4x4} & - \frac{\Delta t}{2}\ \bm{S} \left( \bm{q}_k \right) & \bm{0}_{4x2} \\
		\cdashedline
		\bm{0}_{3x3} & \bm{0}_{3x3} & \bm{0}_{3x4} & \bm{I}_{3x3} & \bm{0}_{3x2} \\
		\cdashedline
		\bm{0}_{2x3} & \bm{0}_{2x3} & \bm{0}_{2x4} & \bm{0}_{2x3} & \bm{I}_{2x2} \\	
	\end{bNiceArray}
	\label{state_jacobi_end}
\end{equation}

Next, a measurement function was determined. The accelerometer reading is used to correct orientation. The external position provider is linked with a position and drawbar orientation. The length of the drawbar is constant and equal $d$. The measurement function is given in equation (\ref{h_fun}). The Jacobi matrix of the measurement function is calculated in equations (\ref{measurement_jacobi_start} - \ref{measurement_jacobi_end}).


\begin{equation}
	\bm{h} \left(\bm{x} \right)
	= 
	\
	\begin{bmatrix}
		\bm{DCM} \left( \bm{q} \right) 
		\begin{bmatrix}
			0 \\ 0 \\ g 
		\end{bmatrix}\\
		\begin{bmatrix}
			x \\ y \\ z 
		\end{bmatrix}
		+
		\bm{R}_z \left( az \right)
		\bm{R}_y \left( el \right)
		\begin{bmatrix}
			d \\ 0 \\ 0 
		\end{bmatrix}
	\end{bmatrix}
	=
	\begin{bmatrix}
		2g(q_x q_z + q_0 q_y) \\
		2g(q_y q_z - q_0 q_x) \\
		g(q_0^2 - q_x^2 - q_y^2 + q_z^2) \\
		x + d\ cos(az)\ cos(el)\\
		y + d\ sin(az)\ cos(el)\\
		z - sin(el)
	\end{bmatrix}
	\label{h_fun}
\end{equation}

\begin{equation}
	\bm{Ca} \left( \bm{q} \right)
	=
	\begin{bmatrix}
		2q_y & 2q_z & q_0 & 2q_x \\
		-2q_x & -2q_0 & 2q_z & 2q_y \\
		2q_0 & -2q_x & -2q_y & 2q_z
	\end{bmatrix}
	\label{measurement_jacobi_start}
\end{equation}

\begin{equation}
	\bm{Cp} \left( az, el\right)
	=
	\begin{bmatrix}
		-sin(az)\ cos(el) & -cos(az)\ sin(el)\\
		 cos(az)\ cos(el) & -sin(az)\ sin(el)\\
		 0				  & -cos(el)
		
	\end{bmatrix}
\end{equation}

\begin{equation}
	\renewcommand{\arraystretch}{1.3}
	\bm{H} \left( \bm{x}, \bm{u} \right) = \frac{\partial \bm{h}}{\partial \bm{x}}
	= 
	\begin{bNiceArray}{c;c;c;c;c}
		\bm{0}_{3x3} & \bm{0}_{3x3} & g\ \bm{Ca} \left( \bm{q} \right)& \bm{0}_{3x3} & \bm{0}_{3x2} \\
		\cdashedline
		\bm{I}_{3x3} & \bm{0}_{3x3} & \bm{0}_{3x4} & \bm{0}_{3x3} & d\ \bm{Cp} \left( az, el\right) \\
	\end{bNiceArray}
	\label{measurement_jacobi_end}
\end{equation}




\section{Time synchronization problem}

Sensor's data flow to the server from multiple sensor's instance with various delays. The reading are used in prediction and correction phases. In correction phase, reading time is no so important, as this phase can be split into multiple correction. Formally, this approach is present in Multiplicative and Sequential Kalman Filters. Unfortunately, in predict phase all involved readings should be synchronized. If time of reading is known, the nearest readings are selected, or if calculation are conducted with fixed step, readings can be interpolated.\\

The time of reading is also not a trivial term, especially in decentralized system. There are many time determination method known, which differ conditions (\cite{time_sync} \cite{time_sync2}). Consider the case that assume precise clock in server and low-precision clock in every sensor node. An example method of solving the synchronization problem presents itself as follows. Every fixed period server broadcasts time synchronize message that contains only unique sequence number. Sending time $T_1$ (according to server's clock) is stored. Every node that receives synchronize message replies immediately with sequence number and node's clock time $T_n$. Received message is saved with timestamp $T_2$. Based on this reply server calculates transmission delay and clocks' offset. First transmission delay (commonly called \textit{ping}) is calculated as a half of a period between broadcasting message and receiving reply (equation (\ref{ping}))). Next, node's clock offset in respect to server's clock is calculated and stored in array (equation (\ref{offset}))). Nodes send readings signed with timestamp, that is based on node's clock. Finally, true timestamp $T_r$ in respect to server's clock is calculated by adding relevant offset (equation (\ref{timestamp}))).

\begin{equation}
	ping = \frac{T_2 - T_1}{2}
	\label{ping}
\end{equation}

\begin{equation}
	offset = T_2 - ping - T_n
	\label{offset}
\end{equation}

\begin{equation}
	T_r = T_n + offset
	\label{timestamp}
\end{equation}