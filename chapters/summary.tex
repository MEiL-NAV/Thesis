\chapter{Summary}

This thesis represents a comprehensive exploration into position and orientation estimation methods. Within the scope of this work it was possible to gather extensive information about sensor fusion, data filtration and signal processing. The thesis introduces a novel approach to utilize a knowledge of the system structure.\\

Based on the knowledge gathered, a working system was successfully designed and deployed. The developed system is an application software that will be used for further research. For this reason, its operation has been further checked and its correctness has been validated.\\

The present work opens the topic of using solutions taken from aviation and other technical sciences in the field of robot positioning. In particular, it is important to highlight the pioneering use of constraints' correction in this field of science.


% This thesis enhances robotic positioning systems by integrating inertial navigation systems, inspired by aviation principles, with knowledge of mechanical design. The work done is part of a project being carried out at the Division of Theory of Machines and Robots. The primary goal is to fill a crucial gap in the existing setup – the absence of position and orientation sensors.\\

% The thesis explores fundamental navigation sensors, recognizing their limitations. It discusses strategies like filtering and sensor fusion to mitigate errors and obtain accurate position and orientation estimates. The proposed solution involves designing the experimental platform. This system modifies inertial navigation to estimate coordinates efficiently as the platform moves, addressing specific motion constraints.\\

% Anticipated outcomes include a deep understanding of position and orientation measurement systems and the design of a prototype tailored for robotic applications. This thesis may improve positioning systems, benefiting both the division's projects and the broader field of robotics.