\section{Introduction}

\subsection{Motivation}

The motivation behind this thesis stems from a division project dedicated to creating hybrid methods for analyzing overconstrained multi-body systems. Research conducted in this project is planned to be validated in an experimental study. The test bench prepared for this purpose has revealed a design flaw -- the absence of position and orientation measuring sensors in the experimental setup. This limitation posed a significant challenge, particularly when attempting to measure forces in dynamic scenarios. The need to correlate position and orientation data with force measurements led to considerations of how measurements can be made in case, the dedicated sensor were not included in system.\\

Selected variant involves usage of inertial sensor and its fusion with sensors commonly used in robotics. The interdisciplinary nature of the project, combining elements from aviation concepts and robotic systems, fueled my enthusiasm to explore and innovate in this unique intersection.

\subsection{The aim of thesis}

The purpose of this project to design universal positioning system based on inertial navigation and knowledge of the multi-body design. The system is an adaptation of Extended Kalman Filter to be used in robotic. To achieve that many sensors are recognized and involved in calculation. The measurements will be filtered and blended to improve estimation result. The algorithm will be optimized to run on a microcontroller in real time. The architecture of system is planned to be very transparent and open to modification in order to increase reusability.\\

The main part of the thesis is a review of the state-of-the-art methods of estimation position and orientation both in aviation and robotic studies. There is many already checked solution that can be partially adopted in the project. Based on gathered information, a novel fusion method will be proposed and check in the simulation and in real deployment.\\

The practical outcome of thesis is a design of prototype able to estimate position and orientation with various sensors set connected. The thesis will include estimation results and error calculation. Tuning and improving scores is also part of the project.