\chapter{Review}

The position and orientation estimation is a problem known in aviation over the years. With the advent of automated control systems, the need to determine exact location and orientation has steadily increased. As a result, mechanical instruments were quickly suppressed by electronic sensors. Based on their measurements, orientation calculation algorithms were developed and implemented. The well-known and used are the Direct-Cosine-Matrix algorithm \cite{dcm}, Complementary filter \cite{complementary} and Madgwick Orientation Filter \cite{madgwick} \cite{Hasan2020}. All of these methods provide tolerable results depending on the set of sensors used, but without a clear mechanism to control particular sensor's involvement.
Also, the position estimation was address by improving path integration methods and inertial sensor measurements integration \cite{farrell2012integrated}. Next, the position was corrected by introducing radio beacon and satellite systems. For a long time, position and orientation were estimated separately.\\

Huge impact on navigation system were caused by adopting the Kalman Filter in estimation. The mathematical concept of a filter based on the equation of state and statistics was first presented in 1960 \cite{kalman}. Since then, the Kalman Filter, especially in its non-linear form (known as the Extended Kalman Filter) became a standard in state estimation and sensor fusion. Its usability was tested in many different scenarios, leading to an extensive database of articles \cite{ekf_poor} \cite{s16020264} \cite{s120709566}. The aviation is not the only application area of the Kalman Filter. Thanks to its versatility, the filter can be used wherever it is possible to arrange the appropriate equations of state.\\

The result of widespread familiarity with the filter is also the development of many modifications and improvements that allow it to be used in specific applications. One of the most useful in presented thesis is possibility to develop the system with an additional correction to meet the constraints of the system. The concept and implementation of correction in article \cite{simon}. Correction leads to exact solution in case of linear equality and inequalities, and gives an estimation in the case of non-linear constraints.\\

The navigation methods outlined also appear in robotics, but the dominant part is mobile robotics \cite{accelerometer_mobile}. In stationary mechanisms, higher accuracy and repeatability are required. For this reason, a common approach is to use encoder readout and position estimation through a forward kinematics task \cite{forward_kinematics}. The significant advantage of this solution is that received results comply with mechanism constraints. In the absence of an encoder or similar built-in sensor (e.g. in biorobotics), computer vision algorithms \cite{cv_positioning} \cite{cv_positioning2} and triangulation methods \cite{igps} are used as a substitute. Inertial sensors are also mounted in industrial robots, but the measurements are usually used in quality and health checks like vibration measures \cite{Dogrusoz_2020}.\\

Every algorithm, no matter how sophisticated, bases its results on the data provided. Poor-quality data leads to errors and high uncertainty of results. Garbage in, garbage out. To prevent this, additional steps should be taken at the system preparation stage. By nature, sensors have finite resolution and sampling time. Each sensor measurement is subject to errors, but many factors contribute to this, like bias or high frequency noise. Same of these factor are inevitable, while others can be strongly reduced, like mounting bias. To improve results, all sensors should be checked and calibrated before usage \cite{mi13060879} \cite{Hol_2011} \cite{gyro_calib}. It is also a good practice to pre-filter raw measurements with frequency filtration \cite{BADRI20101425} to minimize noise and characteristic disturbances. \\

The speed of calculation is an important factor as well. Sensors have various measurement periods and modes. Some of them, especially inertial sensors, are incredibly fast, leaving only a couple of milliseconds for calculation. To achieve good-quality results, as stated in Nyquist–Shannon sampling theorem \cite{sample_theorem}, the frequency of estimation should be at least twice as high as the maximal frequency observed in the mechanism's movement. Achieving high performance on an embedded system requires suitable implementation solutions. Since most of the calculations are conducted on floating precision numbers, hardware acceleration is required  \cite{fpu2} \cite{fpu}.\\

At the time of writing this review, no publication could be found on the application of inertial navigation systems in industrial robotics. This allows us to conclude that the concept defined in the project objective has not yet been studied and is an open research problem.




