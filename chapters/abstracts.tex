% ---------------------------- ABSTRACTS -----------------------------

{  \fontsize{12}{14} \selectfont
	\begin{abstract}
		
		\begin{center}
			\title
		\end{center}
		
		The paper describes the implementation of a system, which is designed to estimate the position and orientation of a robot, based mainly on measurements of inertial sensors. The system allows
		conducting real-time estimation based on diverse data sources. In addition, the system takes into account information about the mechanical design and its constraints. The result of the system's operation is an online visualization and a set of recorded logs, for later analysis.\\
		
		The work gathered and organized knowledge about measurement sensors, methods of filtering their measurements and methods of measurement fusion. Based on the collected information, a Kalman filter model was developed, allowing efficient estimation of the robot's position and orientation. Theoretical considerations were reflected in the actual implementation, the functioning of which was analyzed on a selected test case.\\
		
		\noindent \textbf{Keywords:} estimation, sensors, filtering, sensor fusion, constraints, Extended Kalman filter
	\end{abstract}
}

\null\thispagestyle{empty}\newpage


{\selectlanguage{polish} \fontsize{12}{14}\selectfont
	\begin{abstract}
		
		\begin{center}
			\tytul
		\end{center}
		
		W pracy opisano realizację systemu przeznaczonego do estymacji pozycji i~orientacji robota, bazującego w głownej mierze na pomiarach czujników intercjalych. System pozwala na prowadzenie estymacji w~czasie rzeczywistym w~oparciu o~zróżnicowane źródła danych. Ponadto w~systemie uwzględniona została informacja o konstrukcji mechanicznej i~jej ograniczeniach. Wynikiem działania systemu jest wizualizacja w~czasie rzeczywistym oraz zbiór zarejestrowanych danych, pozwalajacych na późniejszą analizę.\\
		
		W ramach pracy zgromadzono i~uporządkowano wiedzę na temat czujników pomiarowych, metod filtracji ich pomiarów i~metod fuzji pomiarów.
		Bazując na przygotowanych informacjach opracowano model filtru Kalmana, pozwalający na efektywną estymację pozycji i~orientacji robota. Rozważania teoretyczne znalazły odzwiercielenie w rzeczywistej implementacji, której działanie przeanalizowano na wybranym przypadku badawczym.\\
		
		
		\noindent \textbf{Słowa kluczowe:} estymacja, czujniki, filtracja, fuzja czujników, więzy, Rozszerzony filtr Kalmana
	\end{abstract}
}

\null\thispagestyle{empty}\newpage