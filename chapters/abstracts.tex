% ---------------------------- ABSTRACTS -----------------------------

{  \fontsize{12}{14} \selectfont
	\begin{abstract}
		
		\begin{center}
			\title
		\end{center}
		
		The thesis describes an implementation of a system designed to estimate the position and orientation of a mechanism, robot, or similar machine based mainly on inertial sensor measurements. The system allows for conducting real-time estimation based on diverse data sources. Moreover, the system takes into account information regarding the mechanical design and the imposed motion constraints. The system operation results in real-time visualization and a set of recorded logs, allowing for subsequent analysis.\\
		
		The work gathered and systematized knowledge about sensors, methods of filtering their measurements, and techniques of measurement fusion. Based on the collected information, a Kalman filter model was developed, allowing efficient estimation of the mechanism parts' position and
		orientation. An important novelty is the proposal of a modification introduced to the method embedded in the Kalman filter, aiming at taking into account the constraints imposed on the state variables. Theoretical considerations found counterparts in the actual implementation, the functioning of which was analyzed using a selected test case.\\
		
		\noindent \textbf{Keywords:} estimation, sensors, filtering, sensor fusion, constraints, Extended Kalman Filter
	\end{abstract}
}

\null\thispagestyle{empty}\newpage


{\selectlanguage{polish} \fontsize{12}{14}\selectfont
	\begin{abstract}
		
		\begin{center}
			\tytul
		\end{center}
		
		W pracy opisano realizację systemu przeznaczonego do estymacji pozycji i~orientacji mechanizmów, robotów i podobnych maszyn, bazującego w głownej mierze na pomiarach czujników intercjalych. System pozwala na prowadzenie estymacji w~czasie rzeczywistym w~oparciu o~zróżnicowane źródła danych. Ponadto w~systemie uwzględniona została informacja o konstrukcji mechanicznej i~ograniczeniach jej ruchu. Wynikiem działania systemu jest wizualizacja w~czasie rzeczywistym oraz zbiór zarejestrowanych danych, pozwalajacych na późniejszą analizę.\\
		
		W ramach pracy zgromadzono i~uporządkowano wiedzę na temat czujników pomiarowych, metod filtracji ich pomiarów i~metod fuzji pomiarów.
		Bazując na przygotowanych informacjach opracowano model filtru Kalmana, pozwalający na efektywną estymację pozycji i~orientacji członów mechanizmu. Istotną nowością jest zaproponowanie modyfikacji osadzonej w filtrze Kalmana, umożliwiającej uwzglednienie więzów krępujących zmienne stanu. Rozważania teoretyczne znalazły odzwiercielenie w rzeczywistej implementacji, której działanie przeanalizowano na wybranym przypadku badawczym.\\
		
		
		\noindent \textbf{Słowa kluczowe:} estymacja, czujniki, filtracja, fuzja czujników, więzy, rozszerzony filtr Kalmana
	\end{abstract}
}

\null\thispagestyle{empty}\newpage