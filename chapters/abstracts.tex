% ---------------------------- ABSTRACTS -----------------------------

{  \fontsize{12}{14} \selectfont
	\begin{abstract}
		
		\begin{center}
			\title
		\end{center}
		
		\todo[inline, color=green]{
			To fill up, after polish version approvement
		}
		
		\noindent \textbf{Keywords:} keyword1, keyword2, ...
	\end{abstract}
}

\null\thispagestyle{empty}\newpage


{\selectlanguage{polish} \fontsize{12}{14}\selectfont
	\begin{abstract}
		
		\begin{center}
			\tytul
		\end{center}
		
		W pracy opisano realizację systemu przeznaczonego do estymacji pozycji i~orientacji robota, bazując w głownej mierze na pomiarach czujników intercjalych. System pozwala na prowadzenie estymacji w~czasie rzeczywistym w~oparciu o~zróżnicowane źródła danych. Ponadto w~systemie uwzględniona została informacja o konstrukcji mechanicznej i~jej ograniczeniach. Wynikiem działania systemu jest wizualizacja w~czasie rzeczywistym oraz zbiór zarejestrowanych danych, pozwalajacych na późniejszą analizę.\\
		
		W ramach pracy zgromadzono i~uporządkowano wiedzę na temat czujników pomiarowych, metod filtracji ich pomiarów i~metod fuzji pomiarów.
		Bazując na przygotowanych informacjach opracowany został model Filtru Kalmana, pozwalającego na efektywną estymacje pozycji i~orientacji robota. Rozważania teoretyczne znalazły odzwiercielenie w rzeczywistej implementacji, której działanie przeanalizowano na wybranym przypadku badawczym.\\
		
		
		\noindent \textbf{Słowa kluczowe:} estymacja, czujniki, filtracja, fuzja czujników, więzy, Rozszerzony Filtr Kalmana
	\end{abstract}
}

\null\thispagestyle{empty}\newpage