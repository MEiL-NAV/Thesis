\chapter{Sensors}

Sensors are the fundamental components of any measurement system, enabling it to perceive the world. There is a vast array of sensor types, each designed to measure specific physical quantities. Even within a one type of measured value, sensors can vary significantly in terms of precision and cost. \\

Sensors can generally be classified into two main categories: those that operate independently and those that depend on external infrastructure. Independent sensors operate based on the principles of physics. For example, at its simplest, a temperature sensor exploits the fact that electrical resistance changes with temperature variations. On the other hand, dependent sensors require external resources to function. A GNSS receiver, which estimates its position only when a sufficient number of satellites are visible, is an example of such a sensor.\\

Accurately estimating position and orientation typically involves a combination of sensors that measure not only position, velocity, and acceleration but also have an indirect relationship with the state being estimated. The most precise results are achievable when every aspect of the state is directly measured, which is often impractical. The absence of distance sensors and encoders means that state estimation must rely on alternative methods. This thesis will explore dead reckoning, a method that estimates position and orientation by integrating measurements from inertial sensors. At a minimum, this method requires two types of inertial sensors: an accelerometer to measure acceleration and a gyroscope to measure angular velocity. However, results from this minimal setup tend to be suboptimal. In real-world applications, systems are enhanced through redundancy and the incorporation of partial information sources.

\section{Sensor's types}

Sensors measuring a specific type of value, even within the same family, can differ significantly in the methodologies they employ to obtain that value. The development of new sensor types is a constant effort, primarily aimed at enhancing accuracy. However, this pursuit of higher accuracy often results in increased sensor costs. It is worthwhile to briefly outline how these methods have evolved over time.\\

For inertial sensors, this evolution can be traced from mechanical sensors to Micro-Electro-Mechanical Systems (MEMS) sensors, and onto today’s laboratory-grade sensors. Each step on this path not only represents a leap in the technology used but also reflects a balance between achieving greater precision and managing production and operational costs. Mechanical sensors, which were among the first to be developed, rely on mechanical components and physical principles for their operation. Many moving parts often cause that mechanical sensor were unreliable. MEMS sensors, a significant advancement, integrate mechanical components and electronics into a single system, offering improved accuracy, smaller size, and lower power consumption. Today's laboratory-grade sensors, which represent the peak of current technology, provide unparalleled accuracy but at a significantly higher cost. This progression underscores the relentless pursuit of precision in the field of sensor technology, driven by the demands of increasingly sophisticated applications.\\

However, when considering the balance between quality and cost, MEMS sensors are favored in most applications. They guarantee minimal enclosure size and offer quality that is commensurate with their dimensions. Furthermore, the compactness and cost-effectiveness of MEMS sensors make it feasible to deploy arrays of multiple sensors. Integrating their measurements can significantly enhance accuracy. This approach of blending data from multiple MEMS sensors not only improves the precision of measurements but also leverages the strengths of each sensor to compensate for individual limitations, making it a highly efficient strategy in various applications.

\section{Sensor's model}

The description of sensors so far has mainly concerned their features and types. However, to use them in estimation, this description should be enriched with a formalized mathematical model. Sensor's model defines the relation between physical quantities and sensor output, but also includes an error. A well-prepared model allows the development of better filtration methods and ultimately leads to better results.

\subsection{Accelerometer}

An accelerometer measures all accelerations that affect the sensor. It means that if the sensor was mount in a non-inertial reference frame, beside the linear acceleration of frame, it will also measure centripetal, angular and Coriolis acceleration. What's more, masses on Earth experience constant gravitational acceleration, and that feature is especially useful in orientation's estimation. An accelerometer measurement is a resultant of those accelerations. Equation (\ref{acc_model1}) represents measured acceleration as a sum of components. Because of its complexity it is often assume that accelerometer is located in frame center and $\bm{r^W} = \bm{0}$. It leads to simplification given in equation (\ref{acc_model1b}).
\\
\begin{equation}
	\bm{a}_{res}^B = \bm{a}^B + \bm{R}^W_B \left( \bm{\omega}^W \times \left( \bm{\omega}^W \times \bm{r}^W \right) + \bm{\epsilon}^W \times \bm{r}^W - 2\left( \bm{\omega}^W \times \bm{v}^W \right) - \bm{g}  \right)  
	\label{acc_model1}
\end{equation}

\begin{equation}
	\bm{a}_{res}^B = \bm{a}^B - \bm{R}^W_B \left( \bm{g}  \right)  
	\label{acc_model1b}
\end{equation}

where:
\begin{itemize}
	\item $\bm{a}_{res}^B$ -- resultant acceleration given in moving frame (linked with body),
	\item $\bm{a}^B$ -- acceleration of moving frame given in this frame,
	\item $\bm{R}^W_B$ -- rotation matrix form world frame to moving frame,
	\item $\bm{\omega}^W$ -- moving frame's angular velocity given in world frame,
	\item $\bm{\epsilon}^W$ -- moving frame's angular acceleration given in world frame,
	\item $\bm{r}^W$ -- sensor position in moving frame given in world frame,
	\item $\bm{v}^W$ -- moving frame's linear velocity given in world frame,
	\item $\bm{g}$ -- gravitation acceleration in world frame $\left(\begin{bmatrix}
		0 &  0 &  g
	\end{bmatrix}^T \right)$.
	
\end{itemize}

The second part of model is an inclusion of error. Accelerometers' readings are subject to scale error, bias, assembly's inaccuracies and noise.
The scale error means that the readings are linearly dependent of acceleration, but the slope factor is not equal 1 precisely. The bias means that the linear function describing transform is not homogeneous and constant coefficient is non-zero. The assembly inaccuracy is due to the fact that the sensor axes are not collinear with the reference frame. However, the assembly inaccuracy is described as rotation that does not affect the measure's norm. Finally, the noise is a random component added to measure. Fortunately, in case of accelerometer, the noise is negligibly small. Equation (\ref{acc_model2}) describe the correlation between acceleration and output from sensor.\\


\begin{equation}
	\bm{\hat{a}}_{res}^B = \bm{R}_M \bm{S} \left( \bm{a}_{res}^B + \bm{b} \right) + \bm{n}
	\label{acc_model2}
\end{equation}

where:
\begin{itemize}
	\item $\bm{\hat{a}}_{res}^B$ -- sensor output,
	\item $\bm{R}_M$ -- rotation matrix due to assembly inaccuracy,
	\item $\bm{S}$ -- diagonal scale matrix,
	\item $\bm{b}$ -- sensor constant bias,
	\item $\bm{n}$ -- random noise.	
\end{itemize}



\subsection{Gyroscope}

\subsection{External position provider}

\section{Sensor's calibration and filtration}

\subsection{Accelerometer}

\todo[inline, color=green]{
	Mounting rotation, translation. Scaling
}

\subsection{Gyroscope}

\todo[inline, color=green]{
	Bias
}

\subsection{External position provider}

\todo[inline, color=green]{
	Unknown, dependent, only promises
}

